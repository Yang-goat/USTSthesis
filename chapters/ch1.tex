\chapter{模板正文使用说明}\label{chap:usage}

本章用于说明本模板在正文中的常见写作方式,目标是让你可以直接复制示例并替换内容完成论文撰写。建议先通读本章,再按自己的课题逐步替换章节内容。

\section{正文结构与文件组织}

推荐按以下顺序组织内容:
\begin{enumerate}
  \item 在 \verb|main.tex| 中维护论文整体顺序(摘要、目录、正文、结论、参考文献、附录)。
  \item 在 \verb|chapters/| 下按章节拆分正文(如 \verb|ch1.tex|、\verb|ch2.tex|)。
  \item 在 \verb|reference.bib| 中维护参考文献条目,在正文通过 \verb|\cite{key}| 引用。
\end{enumerate}

如果是首次使用,建议先保留模板已有结构,只替换文字和图表内容,不要一次性改动太多排版设置。

\section{交叉引用与 \texttt{cleveref}(推荐)}

本模板已加载 \verb|cleveref|,推荐优先使用 \verb|\cref| 与 \verb|\Cref| 做交叉引用。与 \verb|\ref| 相比,\verb|\cref| 会自动补齐“图、表、式、定理”等前缀,且支持范围引用。

\begin{equation}\label{eq:usage-energy}
  E = mc^2
\end{equation}

\begin{equation}\label{eq:usage-motion}
  s = v_0 t + \frac{1}{2}at^2
\end{equation}

正文中可写为:见\cref{eq:usage-energy};相关运动关系见\cref{eq:usage-motion};连续编号可写为\crefrange{eq:usage-energy}{eq:usage-motion}。

\section{图表写作规范}

\subsection{插图}

\begin{figure}[H]
  \centering
  % 修改图尺寸:把 0.62 改成 0.5/0.7 等;也可改为 height=6cm
  % 常用位置参数:[H] 强制当前位置;[htbp] 允许浮动优化排版
  \includegraphics[width=0.62\textwidth]{example-image-a}
  \caption{单图示例(替换为你的实验图)}
  \label{fig:usage-single}
\end{figure}

\begin{figure}[H]
  \centering
  % 修改子图宽度:0.48\textwidth 适合双栏并排,改为 0.32 可放三图
  \begin{subfigure}{0.48\textwidth}
    \centering
    % 子图内部再调大小:通常保持 width=\textwidth
    \includegraphics[width=\textwidth]{example-image-a}
    \caption{子图A}
    \label{fig:usage-sub-a}
  \end{subfigure}
  \hfill % 子图间距;也可改为 \hspace{1em}
  \begin{subfigure}{0.48\textwidth}
    \centering
    \includegraphics[width=\textwidth]{example-image-b}
    \caption{子图B}
    \label{fig:usage-sub-b}
  \end{subfigure}
  \caption{并列子图示例}
  \label{fig:usage-sub}
\end{figure}

如\crefrange{fig:usage-single}{fig:usage-sub}所示,可分别引用单图、整组子图或单个子图(如\cref{fig:usage-sub-a})。

\subsection{表格}

\begin{table}[H]
  \centering
  \caption{三线表示例}
  \label{tab:usage-basic}
  \begin{tabular}{lll}
    \toprule
    指标 & 方法A & 方法B \\
    \midrule
    准确率 & 91.2\% & 93.4\% \\
    召回率 & 89.8\% & 92.1\% \\
    \bottomrule
  \end{tabular}
\end{table}

\begin{table}[H]
  \centering
  \caption{固定列宽与自动换行示例}
  \label{tab:usage-fixed}
  % 调整行高:1.2 可改为 1.1/1.3
  \renewcommand{\arraystretch}{1.2}
  % 调整列宽:p{3cm} / p{7cm} / p{3cm} 可按内容长度修改
  \begin{tabular}{p{3cm}p{7cm}p{3cm}}
    \toprule
    模块 & 说明 & 备注 \\
    \midrule
    数据预处理 & 对原始样本进行清洗、标准化和格式统一,避免异常值影响训练结果。 & 建议独立成节 \\
    模型训练 & 使用训练集进行参数优化并记录收敛曲线。 & 给出超参数 \\
    \bottomrule
  \end{tabular}
\end{table}

\subsection{长表与续表示例(longtable)}

当表格跨页时,建议使用 \verb|longtable|。第一页显示完整标题,后续页面显示“续表”并重复表头。

\begin{longtable}{p{2cm}p{3cm}p{7cm}}
  \caption{长表续表示例}\label{tab:usage-long}\\
  \toprule
  序号 & 指标 & 说明 \\
  \midrule
  \endfirsthead
  \multicolumn{3}{r}{续表~\ref{tab:usage-long}}\\
  \toprule
  序号 & 指标 & 说明 \\
  \midrule
  \endhead
  \bottomrule
  \endfoot
  \bottomrule
  \endlastfoot
  1 & 指标A & 这是长表正文示例,可按需要替换为真实内容。\\
  2 & 指标B & 该列使用固定宽度 p{7cm},文字会自动换行。\\
  3 & 指标C & 长表可跨页,续页自动重复表头。\\
  4 & 指标D & 可用于参数表、符号表、实验对比明细表。\\
  5 & 指标E & 若内容很多,建议拆分为多张长表并分别引用。\\
  6 & 指标F & 示例数据。\\
  7 & 指标G & 示例数据。\\
  8 & 指标H & 示例数据。\\
  9 & 指标I & 示例数据。\\
  10 & 指标J & 示例数据。\\
  11 & 指标K & 示例数据。\\
  12 & 指标L & 示例数据。\\
  13 & 指标M & 示例数据。\\
  14 & 指标N & 示例数据。\\
  15 & 指标O & 示例数据。\\
  16 & 指标P & 示例数据。\\
  17 & 指标Q & 示例数据。\\
  18 & 指标R & 示例数据。\\
  19 & 指标S & 示例数据。\\
  20 & 指标T & 示例数据。\\
  21 & 指标U & 示例数据。\\
  22 & 指标V & 示例数据。\\
  23 & 指标W & 示例数据。\\
  24 & 指标X & 示例数据。\\
  25 & 指标Y & 示例数据。\\
  26 & 指标Z & 示例数据。\\
  27 & 指标AA & 示例数据。\\
  28 & 指标AB & 示例数据。\\
  29 & 指标AC & 示例数据。\\
  30 & 指标AD & 示例数据。\\
\end{longtable}

\subsection{长表分组小计与总计示例}

当结果需要按组展示并给出小计、总计时,可在 \verb|longtable| 内插入分组行。下面示例可直接改为你的统计表:

\begin{longtable}{p{2cm}p{4cm}p{2cm}p{2cm}p{2cm}}
  \caption{长表分组统计(含小计与总计)}\label{tab:usage-long-sum}\\
  \toprule
  组别 & 项目 & 数量 & 均值 & 备注 \\
  \midrule
  \endfirsthead
  \multicolumn{5}{r}{续表~\ref{tab:usage-long-sum}}\\
  \toprule
  组别 & 项目 & 数量 & 均值 & 备注 \\
  \midrule
  \endhead
  \bottomrule
  \endfoot
  \bottomrule
  \endlastfoot
  \multicolumn{5}{l}{\textbf{A组}}\\
  A & 指标A1 & 12 & 91.2 & 示例数据\\
  A & 指标A2 & 15 & 90.4 & 示例数据\\
  A & 指标A3 & 11 & 92.1 & 示例数据\\
  \midrule
  \multicolumn{2}{r}{\textbf{A组小计}} & \textbf{38} & \textbf{91.2} & \\
  \midrule
  \multicolumn{5}{l}{\textbf{B组}}\\
  B & 指标B1 & 10 & 88.3 & 示例数据\\
  B & 指标B2 & 14 & 89.1 & 示例数据\\
  B & 指标B3 & 16 & 90.0 & 示例数据\\
  \midrule
  \multicolumn{2}{r}{\textbf{B组小计}} & \textbf{40} & \textbf{89.2} & \\
  \midrule
  \multicolumn{2}{r}{\textbf{总计}} & \textbf{78} & \textbf{90.2} & \textbf{可改为加权均值}\\
\end{longtable}

正文中建议使用\crefrange{tab:usage-basic}{tab:usage-fixed}、\cref{tab:usage-long}与\cref{tab:usage-long-sum}进行说明,并在文字中解释“表格结论”,避免只贴表不分析。

\section{列表环境:\texttt{enumerate}、\texttt{itemize}、\texttt{description}}

\subsection{有序列表(步骤类)}
\begin{enumerate}
  \item 采集原始数据并完成清洗。
  \item 划分训练集、验证集和测试集。
  \item 训练模型并记录关键指标。
\end{enumerate}

\subsection{无序列表(要点类)}
\begin{itemize}
  \item 强调贡献点:突出你的方法与现有工作的差异。
  \item 强调可复现性:给出关键参数、实验设置、评价指标。
  \item 强调结论边界:说明方法在何种条件下有效。
\end{itemize}

\subsection{描述列表(术语解释类)}
\begin{description}
  \item[训练集] 用于模型参数学习的数据集合。
  \item[验证集] 用于调参与模型选择,不参与最终测试统计。
  \item[测试集] 用于报告最终泛化性能的数据集合。
\end{description}

\section{定理、定义与证明环境}

本模板已预置 \verb|theorem|、\verb|definition|、\verb|lemma|、\verb|proposition|、\verb|corollary| 环境,可直接使用。

\begin{definition}\label{def:metric-space}
设非空集合 $X$ 上的函数 $d: X\times X\to \mathbb{R}$ 满足非负性、对称性和三角不等式,则称 $(X,d)$ 为度量空间。
\end{definition}

\begin{theorem}\label{thm:sample}
在度量空间中,收敛序列的极限唯一。
\end{theorem}

\begin{proofenv}
设序列 $\{x_n\}$ 同时收敛到 $x$ 与 $y$。由三角不等式有
\[
  d(x,y)\le d(x,x_n)+d(x_n,y).
\]
令 $n\to\infty$,右侧趋于 $0$,故 $d(x,y)=0$,即 $x=y$。
\end{proofenv}

在正文中可写:由\cref{def:metric-space}可得基本结构;关键结论见\cref{thm:sample}。

\section{公式与算法环境}

\subsection{公式排版建议}

\begin{align}
  f(x) &= x^2 + 2x + 1 \\
       &= (x+1)^2
\end{align}

多行推导建议使用 \verb|align| 或 \verb|aligned|;需要编号时使用带星号以外的环境(如 \verb|align| 而非 \verb|align*|)。

\subsubsection*{1) 多行推导与单编号拆分}
当一组推导需要对齐等号时,使用 \verb|align|;当你希望“多行但只有一个编号”时,使用 \verb|equation+split|:
\begin{equation}\label{eq:usage-split}
  \begin{split}
    \int_0^1 (ax+b)^2\,dx
      &= \int_0^1 (a^2x^2+2abx+b^2)\,dx \\
      &= \frac{a^2}{3}+ab+b^2 .
  \end{split}
\end{equation}

\subsubsection*{2) 子编号公式组(a,b,c)}
相关公式可使用 \verb|subequations| 形成主编号下的子编号:
\begin{subequations}\label{eq:usage-ode}
  \begin{align}
    y_1'(t) &= a_{11}y_1+a_{12}y_2, \label{eq:usage-ode-a}\\
    y_2'(t) &= a_{21}y_1+a_{22}y_2, \label{eq:usage-ode-b}\\
    \bm{y}'(t) &= A\bm{y}(t). \label{eq:usage-ode-c}
  \end{align}
\end{subequations}
正文可写为:见\crefrange{eq:usage-ode-a}{eq:usage-ode-c}。

\subsubsection*{3) 分段函数、极限与定积分}
\begin{equation}\label{eq:usage-cases}
  g(x)=
  \begin{cases}
    x^2\sin\frac{1}{x}, & x\neq 0,\\
    0, & x=0.
  \end{cases}
\end{equation}
\begin{equation}\label{eq:usage-limit}
  \lim_{n\to\infty}\left(1+\frac{x}{n}\right)^n = e^x,\qquad
  \int_{-\infty}^{+\infty} e^{-x^2}\,dx=\sqrt{\pi}.
\end{equation}

\subsubsection*{4) 矩阵、行列式与分块矩阵}
\begin{equation}\label{eq:usage-matrix}
  A=
  \begin{bmatrix}
    1 & 2 & 0\\
    0 & 3 & -1\\
    2 & 1 & 4
  \end{bmatrix},\quad
  \det(A)=
  \begin{vmatrix}
    1 & 2 & 0\\
    0 & 3 & -1\\
    2 & 1 & 4
  \end{vmatrix}.
\end{equation}
\begin{equation}\label{eq:usage-block}
  M=
  \begin{bmatrix}
    B & C\\
    D & E
  \end{bmatrix},\qquad
  B\in\mathbb{R}^{m\times m},\ C\in\mathbb{R}^{m\times n}.
\end{equation}

\subsubsection*{5) 方程组与大括号}
\begin{equation}\label{eq:usage-system}
  \left\{
  \begin{aligned}
    x+y+z &= 1,\\
    2x-y+3z &= 0,\\
    x-4y+z &= 2.
  \end{aligned}
  \right.
\end{equation}

\subsubsection*{6) 优化问题标准写法}
\begin{equation}\label{eq:usage-opt-problem}
  \begin{aligned}
    \min_{x\in\mathbb{R}^n}\quad & f(x)=\frac{1}{2}x^\top Qx+c^\top x\\
    \text{s.t.}\quad & Ax\le b,\quad Ex=d .
  \end{aligned}
\end{equation}
若需给出一阶必要条件,可继续写为:
\begin{equation}\label{eq:usage-kkt}
  \nabla f(x^\star)+A^\top\lambda^\star+E^\top\nu^\star=0,\quad
  \lambda^\star\ge 0,\quad
  \lambda_i^\star (a_i^\top x^\star-b_i)=0.
\end{equation}

\subsubsection*{7) 常用数学符号排版建议}
\begin{itemize}
  \item 向量、矩阵建议统一风格:如 \verb|\bm{x}|、\verb|\bm{A}|。
  \item 使用 \verb|\left...\right| 处理自适应括号,避免手动调大小。
  \item 范数、绝对值可写作 \verb|\lVert x\rVert_2|、\verb||x||。
  \item 重要公式务必加 \verb|\label|,正文用 \verb|\cref| 引用,如\cref{eq:usage-split,eq:usage-opt-problem,eq:usage-kkt}。
\end{itemize}

\subsection{算法示例}

\begin{algorithm}[H]
\caption{示例算法:迭代优化}
\label{alg:usage-opt}
\KwIn{初始解 $x_0$,最大迭代次数 $T$}
\KwOut{近似最优解 $x^\star$}
$x \leftarrow x_0$;\\
\For{$t=1$ \KwTo $T$}{
  计算目标函数与梯度;\\
  按更新规则修正 $x$;\\
}
\KwRet{$x^\star$}
\end{algorithm}

算法引用可写为:核心流程见\cref{alg:usage-opt}。

\section{代码与参考文献引用}

\subsection{代码片段}
\begin{lstlisting}[language=Python,caption={示例:训练循环伪代码},label={lst:usage-train}]
for epoch in range(max_epoch):
    train_loss = train_one_epoch(model, train_loader)
    val_metric = evaluate(model, val_loader)
    if val_metric > best_metric:
        best_metric = val_metric
        save_checkpoint(model)
\end{lstlisting}

\subsection{文献引用}
正文中可直接使用 \verb|\cite{key}|。例如,相关综述可参考\cite{alorf_survey_2023}。编译顺序建议为:
\begin{enumerate}
  \item \verb|xelatex main|
  \item \verb|biber main|
  \item \verb|xelatex main|
  \item \verb|xelatex main|
\end{enumerate}

\section{写作检查清单(提交前)}

\begin{itemize}
  \item 图、表、公式、算法是否都带有 \verb|\caption| 与 \verb|\label|。
  \item 是否统一使用 \verb|\cref| 引用,避免“见下图/上表”这类模糊表述。
  \item 每个图表后是否有分析文字,不只给出结果。
  \item 参考文献是否完整、是否与正文引用一一对应。
\end{itemize}
