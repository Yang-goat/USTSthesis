\chapter{写作与排版示例}\label{chap:usage}

本章演示常见的写作元素用法,包括图片与子图、表格与列宽控制、交叉引用与参考文献,示例内容可直接复制修改。

\section{插入图片与子图}
% 使用 graphicx 宏包的 \includegraphics 插图。
% 常用尺寸控制:
% - width=0.6\textwidth  按页面正文宽度的 60% 缩放(最推荐)
% - height=6cm           指定高度;可与 keepaspectratio 搭配避免变形
% - scale=0.8            相对原始大小缩放
% 位置控制参数([H]/[htbp]):H=强制此处;htbp=建议放置位置,按此顺序尝试。
\begin{figure}[H]
  \centering
  % 将 example-image-a 替换为你的图片文件名(支持 .pdf/.png/.jpg 等)
  % 建议将图片放在 figures/ 目录,并使用相对路径:figures/xxx.png
  \includegraphics[width=0.6\textwidth]{example-image-a} % width 可改为 0.7\textwidth 等
  \caption{单幅图片示例(请替换为真实图题)}
  \label{fig:single}
\end{figure}

% 子图示例:使用 subcaption 宏包提供的 subfigure 环境
% 子图常用设置:
% - 每个子图可单独设置宽度,例如 0.45\textwidth 两列并排
% - 子图标签 \label 放在对应 \caption 之后便于交叉引用
\begin{figure}[H]
  \centering
  \begin{subfigure}{0.48\textwidth}
    \centering
    \includegraphics[width=\textwidth]{example-image-a}
    \caption{子图 A}
    \label{fig:sub-a}
  \end{subfigure}
  \hfill
  \begin{subfigure}{0.48\textwidth}
    \centering
    \includegraphics[width=\textwidth]{example-image-b}
    \caption{子图 B}
    \label{fig:sub-b}
  \end{subfigure}
  \caption{并列子图示例}
  \label{fig:subfig}
\end{figure}

在文中引用图如图~\ref{fig:single} 与 \ref{fig:subfig} 所示;子图可引用为图~\ref{fig:sub-a}、图~\ref{fig:sub-b}。

\section{表格与列宽控制}
% 基础表格:使用 booktabs 提供的 \toprule/\midrule/\bottomrule,排版更美观。
\begin{table}[H]
  \centering
  \caption{基础表格示例}
  \label{tab:basic}
  \begin{tabular}{lll} % l/c/r 分别表示左/居中/右对齐
    \toprule
    列1 & 列2 & 列3 \\
    \midrule
    A1 & B1 & C1 \\
    A2 & B2 & C2 \\
    \bottomrule
  \end{tabular}
\end{table}

% 固定列宽:使用 p{<宽度>} 指定列宽;需根据内容调整宽度值(如 3cm/6em/0.25\textwidth)。
% 可结合 \arraystretch 调整行高:\renewcommand{\arraystretch}{1.2}
\begin{table}[H]
  \centering
  \caption{固定列宽(p{..})示例}
  \label{tab:fixed}
  \renewcommand{\arraystretch}{1.2}
  \begin{tabular}{p{3cm} p{5cm} p{4cm}}
    \toprule
    指标 & 描述 & 备注 \\
    \midrule
    指标一 & 这一列限制为 5cm,文字会自动换行,避免超出页面。 & 可在此添加简短说明。 \\
    指标二 & 根据内容调整列宽以获得更好排版效果。 & 建议避免过长的未换行英文串。 \\
    \bottomrule
  \end{tabular}
\end{table}

% 自适应列宽:使用 tabularx 让表格总宽度填满 \textwidth,并使用 X 列型自动分配宽度。
% 注意:需要 \usepackage{tabularx}(模板已加载)。
\begin{table}[H]
  \centering
  \caption{自适应列宽(tabularx)示例}
  \label{tab:tabularx}
  \begin{tabularx}{\textwidth}{lXX}
    \toprule
    项 & 说明(X 列自动拉伸) & 备注(X 列自动拉伸) \\
    \midrule
    A & 使用 X 列可避免手工调列宽 & 与 p{..} 方案互补,按需选择 \\
    B & 可结合 \verb|\arraystretch| 调整行高 & 复杂表格可考虑 longtable \\
    \bottomrule
  \end{tabularx}
\end{table}

如表~\ref{tab:basic}–\ref{tab:tabularx} 所示,按需要选择基础、固定宽度或自适应宽度的表格方式。

\subsection{长表格与续表格式}
当表格内容较长跨页时,建议使用 longtable 环境实现“续表”。续表时按规范:表题省略、表头重复,并在右上方标注“续表××”。示例如表~\ref{tab:lt} 所示。

% 注意:longtable 的表头/续表头分别由 \endfirsthead 与 \endhead 分隔
\begin{longtable}{p{2.2cm} p{3.2cm} p{6.5cm}}
  \caption{长表格与续表示例}\label{tab:lt}\\
  \toprule
  序号 & 指标 & 说明(可自动换行) \\
  \midrule
  \endfirsthead
  % —— 续表页眉(不重复表题),右上角“续表××” ——
  \multicolumn{3}{r}{续表~\ref{tab:lt}}\\
  \toprule
  序号 & 指标 & 说明(可自动换行) \\
  \midrule
  \endhead
  % —— 表尾(每页) —— 可按需加入“续下页”提示 ——
  \bottomrule
  \endfoot
  % —— 最终表尾(仅最后一页) ——
  \bottomrule
  \endlastfoot
  % ===== 表体示例若干行,保证跨页 =====
  1  & 指标 A & 用于演示 longtable 的跨页续表用法。 \\
  2  & 指标 B & 表头会在续表页重复;续表页不再出现表题,仅在右上角显示“续表××”。 \\
  3  & 指标 C & 可以使用中文和英文混排,长文本会自动换行以适应列宽。 \\
  4  & 指标 D & 本模板使用 booktabs 的横线命令,排版更为美观。 \\
  5  & 指标 E & 若需要脚注或注释,可在表格外另行给出。 \\
  6  & 指标 F & 示例行。 \\
  7  & 指标 G & 示例行。 \\
  8  & 指标 H & 示例行。 \\
  9  & 指标 I & 示例行。 \\
  10 & 指标 J & 示例行。 \\
  11 & 指标 K & 示例行。 \\
  12 & 指标 L & 示例行。 \\
  13 & 指标 M & 示例行。 \\
  14 & 指标 N & 示例行。 \\
  15 & 指标 O & 示例行。 \\
  16 & 指标 P & 示例行。 \\
  17 & 指标 Q & 示例行。 \\
  18 & 指标 R & 示例行。 \\
  19 & 指标 S & 示例行。 \\
  20 & 指标 T & 示例行。 \\
  21 & 指标 U & 示例行。 \\
  22 & 指标 V & 示例行。 \\
  23 & 指标 W & 示例行。 \\
  24 & 指标 X & 示例行。 \\
  25 & 指标 Y & 示例行。 \\
  26 & 指标 Z & 示例行。 \\
  27 & 指标 AA & 示例行。 \\
  28 & 指标 AB & 示例行。 \\
  29 & 指标 AC & 示例行。 \\
  30 & 指标 AD & 示例行。 \\
  31 & 指标 AE & 示例行。 \\
  32 & 指标 AF & 示例行。 \\
  33 & 指标 AG & 示例行。 \\
  34 & 指标 AH & 示例行。 \\
  35 & 指标 AI & 示例行。 \\
  36 & 指标 AJ & 示例行。 \\
  37 & 指标 AK & 示例行。 \\
  38 & 指标 AL & 示例行。 \\
  39 & 指标 AM & 示例行。 \\
  40 & 指标 AN & 示例行。 \\
  41 & 指标 AO & 示例行。 \\
  42 & 指标 AP & 示例行。 \\
  43 & 指标 AQ & 示例行。 \\
  44 & 指标 AR & 示例行。 \\
  45 & 指标 AS & 示例行。 \\
  46 & 指标 AT & 示例行。 \\
  47 & 指标 AU & 示例行。 \\
  48 & 指标 AV & 示例行。 \\
  49 & 指标 AW & 示例行。 \\
  50 & 指标 AX & 示例行(跨页应出现“续表~\ref{tab:lt}”且表头重复)。 \\
\end{longtable}

\section{公式与参考文献}

\subsection{对齐与多行公式(align/split)}
常见等式组或推导可用 \verb|align| 对齐等号:
\begin{align}
  f(x)
    &= x^3 - 3x + 1 \\
    &= (x-1)(x^2 + x - 1) \\
    &= (x-1)\Bigl(x - \tfrac{-1+\sqrt{5}}{2}\Bigr)\Bigl(x - \tfrac{-1-\sqrt{5}}{2}\Bigr)\,.
\end{align}
若只想对单个编号使用多行,可用 \verb|split|:
\begin{equation}\label{eq:split-demo}
  \begin{split}
    y 
      &= ax^2 + bx + c \\
      &= a\Bigl(x + \tfrac{b}{2a}\Bigr)^2 - \tfrac{b^2}{4a} + c\,.
  \end{split}
\end{equation}

\subsection{子编号(a,b,c)—— subequations}
一组相关公式可共享同一主编号并附 a/b/c 子编号:
\begin{equation}\label{eq:set}
  S = \{\, x \in \mathbb{R} \mid x>0 \,\}
\end{equation}
\begin{subequations}\label{eq:abc}
  \begin{align}
    F_a(x) &= \int_0^x e^{-t}\,dt, \label{eq:abc-a}\\
    F_b(x) &= \sum_{k=0}^{\infty} \frac{x^k}{k!}, \label{eq:abc-b}\\
    F_c(x) &= \lim_{n\to\infty} \Bigl(1+\tfrac{x}{n}\Bigr)^n. \label{eq:abc-c}
  \end{align}
\end{subequations}
其中式~\eqref{eq:abc-a}–\eqref{eq:abc-c} 采用 (\ref{eq:abc}) 的子编号。

\subsection{花括号与分段函数(cases/underbrace)}
使用 \verb|cases| 可排版分段函数,并配合左大括号:
\begin{equation}
  f(x) = \begin{cases}
    x^2, & x\ge 0,\\
    -x,  & x<0.
  \end{cases}
\end{equation}
也可用下花括号标注分组:
\begin{equation}
  \underbrace{a + a + \cdots + a}_{n\text{ 项}} = na\,.
\end{equation}

\subsection{矩阵与向量}
常用矩阵环境包括 \verb|pmatrix|、\verb|bmatrix| 与 \verb|vmatrix|:
\begin{equation}
  A = \begin{bmatrix}
    1 & 0 & 2 \\
    -1 & 3 & 1 \\
    0 & 4 & -2
  \end{bmatrix},\quad
  \vec{v} = \begin{pmatrix} 1 \\ 2 \\ 3 \end{pmatrix},\quad
  \det(A) = \begin{vmatrix}
    1 & 0 & 2 \\
    -1 & 3 & 1 \\
    0 & 4 & -2
  \end{vmatrix}.
\end{equation}

\subsection{带左大括号的方程组}
可用 \verb|aligned| 搭配可伸缩括号:
\begin{equation}
  \left\{ \begin{aligned}
    x + y &= 1, \\
    2x - y &= 0.
  \end{aligned} \right.
\end{equation}
% 公式写作建议:使用 \label 为重要公式编号并可引用;复杂公式使用 align/align* 环境。
\begin{equation}\label{eq:motion}
  s = v_0 t + \tfrac{1}{2} a t^2
\end{equation}
式~\eqref{eq:motion} 给出匀加速直线运动的位移计算。

% 参考文献:参考文献条目在 reference.bib 中;在文中使用 \cite{key} 引用。
例如,参见文献~\cite{alorf_survey_2023} 获取更多综述性介绍。
